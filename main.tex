\documentclass[coursework]{SCWorks}
% Тип обучения (одно из значений):
%    bachelor   - бакалавриат (по умолчанию)
%    spec       - специальность
%    master     - магистратура
% Форма обучения (одно из значений):
%    och        - очное (по умолчанию)
%    zaoch      - заочное
% Тип работы (одно из значений):
%    coursework - курсовая работа (по умолчанию)
%    referat    - реферат
%  * otchet     - универсальный отчет
%  * nirjournal - журнал НИР
%  * digital    - итоговая работа для цифровой кафедры
%    diploma    - дипломная работа
%    pract      - отчет о научно-исследовательской работе
%    autoref    - автореферат выпускной работы
%    assignment - задание на выпускную квалификационную работу
%    review     - отзыв руководителя
%    critique   - рецензия на выпускную работу

% * Добавлены вручную. За вопросами к @mchernigin
\usepackage{preamble}

\begin{document}

% Кафедра (в родительном падеже)
\chair{математической кибернетики и компьютерных наук}

% Тема работы
\title{Разработка ядра клиент-серверного приложения для "Отработки" для андроида}

% Курс
\course{2}

% Группа
\group{251}

% Факультет (в родительном падеже) (по умолчанию "факультета КНиИТ")
% \department{факультета КНиИТ}

\napravlenie{09.03.04 "--- Программная инженерия}

% Для студентки. Для работы студента следующая команда не нужна.
% \studenttitle{студентки}

% Фамилия, имя, отчество в родительном падеже
\author{Смирнова Егора Ильича}

% Заведующий кафедрой 
\chtitle{доцент, к.\,ф.-м.\,н.}
\chname{С.\,В.\,Миронов}

% Руководитель ДПП ПП для цифровой кафедры (перекрывает заведующего кафедры)
% \chpretitle{
%     заведующий кафедрой математических основ информатики и олимпиадного\\
%     программирования на базе МАОУ <<Ф"=Т лицей №1>>
% }
% \chtitle{г. Саратов, к.\,ф.-м.\,н., доцент}
% \chname{Кондратова\, Ю.\,Н.}

% Научный руководитель (для реферата преподаватель проверяющий работу)
\satitle{доцент, к.\,ф.-м.\,н.} %должность, степень, звание
\saname{И.\,А.\,Батраева}

% Руководитель практики от организации (руководитель для цифровой кафедры)
% \patitle{доцент, к.\,ф.-м.\,н.}
% \paname{С.\,В.\,Миронов}

% Руководитель НИР
% \nirtitle{доцент, к.\,п.\,н.} % степень, звание
% \nirname{В.\,А.\,Векслер}

% Семестр (только для практики, для остальных типов работ не используется)
\term{2}

% Наименование практики (только для практики, для остальных типов работ не
% используется)
% \practtype{учебная}

% Продолжительность практики (количество недель) (только для практики, для
% остальных типов работ не используется)
% \duration{2}

% Даты начала и окончания практики (только для практики, для остальных типов
% работ не используется)
% \practStart{01.07.2024}
% \practFinish{13.01.2024}

% Год выполнения отчета
\date{2025}

\maketitle

% Включение нумерации рисунков, формул и таблиц по разделам (по умолчанию -
% нумерация сквозная) (допускается оба вида нумерации)
% \secNumbering

\tableofcontents

% Раздел "Обозначения и сокращения". Может отсутствовать в работе
% \abbreviations
% \begin{description}
%     \item ... "--- ...
%     \item ... "--- ...
% \end{description}

% Раздел "Определения". Может отсутствовать в работе
% \definitions

% Раздел "Определения, обозначения и сокращения". Может отсутствовать в работе.
% Если присутствует, то заменяет собой разделы "Обозначения и сокращения" и
% "Определения"
% \defabbr

\intro
Цель работы --- разработка серверной части системы для приложения "Отработки" с REST API на .NET.

Задачи работы:
\begin{itemize}
	\item{Изучить принципы разработки серверных приложений на .NET.}
	\item{Спроектировать архитектуру серверной части.}
	\item{Реализовать API для управления пользовтелями, запросами на отработку.}
\end{itemize}

% После введения — серии \section, \subsection и т.д.

\conclusion
В рамках проделанной работы была с помощью REST API на .NET реализована серверная часть системы, позволяющей преподавателям контролировать участие студентов в различных мероприятиях.

Система реализует ленту запросов, возможность записи и отписки с запроса. Была разработана система очков, получаемых студентов по результату участия в мероприятии, а также система логирования и ролевая модель, которая предоставляет возможность повышения и понижения участников.

Процесс аутентификации и авторизации был реализован с применением хэширования BCrypt, что делает эти процессы безопаснее.

% Отобразить все источники. Даже те, на которые нет ссылок.
\nocite{*}

\bibliographystyle{ugost2003}
\bibliography{thesis}

% Окончание основного документа и начало приложений Каждая последующая секция
% документа будет являться приложением
\appendix
\section{Полный код TokenServices}\label{app:TokenServices}
\begin{minted}{cs}
using System.IdentityModel.Tokens.Jwt;
using Microsoft.IdentityModel.Tokens;
using System.Security.Claims;
using System.Text;
using DataBase;
using Context;
namespace Services;

public class TokenServices : ITokenServices
{
    private readonly IRefreshTokenRepository _RefreshTokenRepository;
    private readonly int accessTokenTime = 5;
    private readonly int refreshTokenTime = 1440;

    public TokenServices(IRefreshTokenRepository refreshTokenRepository)
    {
        _RefreshTokenRepository = refreshTokenRepository;
    }

    public async Task<Tokens> GenerateJWTToken(User user, string secretKey)
    {
        Tokens jwtToken = new Tokens();

        string accessToken = GenerateAccessToken(user, secretKey);
        string refreshToken = GenerateRefreshToken(user, secretKey);

        jwtToken.AccessToken = accessToken;
        jwtToken.RefreshToken = refreshToken;

        RefreshToken saveRefreshToken = new RefreshToken();
        saveRefreshToken.Id = user.Id;
        saveRefreshToken.Token = refreshToken;

        // Находим oldRefreshToken
        var oldRefreshDB = await _RefreshTokenRepository.FirstOrDefaultAsync(token => token.Id == saveRefreshToken.Id);
        // Удаляем oldRefreshToken, если тот есть
        if (oldRefreshDB != null)
        {
            await _RefreshTokenRepository.Delete(oldRefreshDB);

        }
        await _RefreshTokenRepository.Create(saveRefreshToken);

        return jwtToken;
    }

    public async Task<IBaseResponse<Tokens>> RefreshToken(string oldRefreshToken, string secretKey)
    {
        BaseResponse<Tokens> response = ValidateRefreshToken(oldRefreshToken, secretKey);

        // Если oldRefreshToken невалидный
        if (response.StatusCode == StatusCodes.BadRequest)
        {
            return response;
        }

        var oldToken = new JwtSecurityTokenHandler().ReadJwtToken(oldRefreshToken);
        User user = new User();
        // Заполняем user на основе данных из oldRefreshToken
        user.Id = Convert.ToInt32(oldToken.Claims.First(
                    claim => claim.Type == JwtRegisteredClaimNames.Sub
                    ).Value);
        user.Name = Convert.ToString(oldToken.Claims.First(
                    claim => claim.Type == JwtRegisteredClaimNames.Name
                    ).Value);
        user.Role = Convert.ToString(oldToken.Claims.First(
                    claim => claim.Type == ClaimTypes.Role
                    ).Value);

        // Находим oldRefreshToken
        var oldRefreshDB = await _RefreshTokenRepository.FirstOrDefaultAsync(token => token.Token == oldRefreshToken);

        if (oldRefreshDB == null)
        {
            response = BaseResponse<Tokens>.NotFound();
            return response;
        }

        var oldTokenDB = new JwtSecurityTokenHandler().ReadJwtToken(oldRefreshDB.Token);


        // Оставляем старый refreshToken, или в случае просрока заменим далее
        string refreshToken = oldRefreshDB.Token;
        if (response.Description == "Token expired")
        {
            // Удаляем oldRefreshToken, если тот истек
            await _RefreshTokenRepository.Delete(oldRefreshDB);
            // Пересоздаем oldRefreshToken, если тот истек
            refreshToken = GenerateRefreshToken(user, secretKey);

            RefreshToken saveRefreshToken = new RefreshToken
            {
                Id = user.Id,
                Token = refreshToken
            };
            // Сохраняем новый refreshToken в бд
            await _RefreshTokenRepository.Create(saveRefreshToken);
        }

        string accessToken = GenerateAccessToken(user, secretKey);
        response = BaseResponse<Tokens>.Ok(new Tokens
        {
            AccessToken = accessToken,
            RefreshToken = refreshToken
        });
        return response;
    }

    public async Task<IBaseResponse> DeleteRefreshToken(int userId)
    {
        BaseResponse response;

        var refreshToken = await _RefreshTokenRepository.FirstOrDefaultAsync(token => token.Id == userId);

        if (refreshToken == null)
        {
            response = BaseResponse.NoContent();
            return response;
        }
        await _RefreshTokenRepository.Delete(refreshToken);

        response = BaseResponse.Ok();
        return response;
    }

    private string GenerateAccessToken(User user, string secretKey)
    {
        var claims = GenerateClaimsAccess(user);
        var token = GenerateToken(claims, secretKey, accessTokenTime);
        return new JwtSecurityTokenHandler().WriteToken(token);
    }

    private string GenerateRefreshToken(User user, string secretKey)
    {
        var claims = GenerateClaimsRefresh(user);
        var token = GenerateToken(claims, secretKey + "sault", refreshTokenTime);
        return new JwtSecurityTokenHandler().WriteToken(token);
    }

    private JwtSecurityToken GenerateToken(Claim[] claims, string secretKey, int time)
    {
        var key = new SymmetricSecurityKey(Encoding.UTF8.GetBytes(secretKey));
        var creds = new SigningCredentials(key, SecurityAlgorithms.HmacSha256);

        var token = new JwtSecurityToken(
                issuer: "axecac-kek.ru",
                audience: "axecac-kek.ru",
                claims: claims,
                expires: DateTime.Now.AddMinutes(time),
                signingCredentials: creds);

        return token;
    }

    private Claim[] GenerateClaimsAccess(User user)
    {
        var claims = new[]
        {
            new Claim(JwtRegisteredClaimNames.Name, user.Name),
            new Claim(JwtRegisteredClaimNames.Sub, user.Id.ToString()),
            new Claim(ClaimTypes.Role, user.Role),
        };
        return claims;
    }

    private Claim[] GenerateClaimsRefresh(User user)
    {
        var claims = new[]
        {
            new Claim(JwtRegisteredClaimNames.Name, user.Name),
            new Claim(JwtRegisteredClaimNames.Sub, user.Id.ToString()),
            new Claim(JwtRegisteredClaimNames.Jti, Guid.NewGuid().ToString()),
            new Claim(ClaimTypes.Role, user.Role),
        };
        return claims;
    }

    private BaseResponse<Tokens> ValidateRefreshToken(string token, string secretKey)
    {
        var tokenHandler = new JwtSecurityTokenHandler();
        var validationParameters = new TokenValidationParameters
        {
            ValidateIssuerSigningKey = true,
            IssuerSigningKey = new SymmetricSecurityKey(Encoding.ASCII.GetBytes(secretKey + "sault")),
            ValidateIssuer = false,
            ValidateAudience = false,
            ClockSkew = TimeSpan.Zero
        };
        try
        {
            // Проверка подписи JWT, если токен не верный будет исключение
            tokenHandler.ValidateToken(token, validationParameters, out _);
            return BaseResponse<Tokens>.Ok();
        }
        catch (SecurityTokenExpiredException ex)
        {
            return BaseResponse<Tokens>.Ok(description: "Token expired");
        }
        catch (SecurityTokenException ex)
        {
            return BaseResponse<Tokens>.BadRequest(description: "Tokens not valid");
        }
    }
}
\end{minted}


\section{Архив с окончательным вариантом решения}\label{app:Others}

В приложенном флеш накопителе, для ознакомления, содержатся файлы и код решения и файл данной работы в pdf формате.


\end{document}
