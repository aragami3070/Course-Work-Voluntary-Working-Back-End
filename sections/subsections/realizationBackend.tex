\subsection{Реализция серверной части}

\subsubsection{Структура решения (Сущности, репозитории, сервисы, контроллеры(не уверен, что надо, т.к. они довольно однообразны))}

Пока не знаю как это вменяемо оформить.

\subsubsection{Реализация JWT}
% NOTE: JWT
Данная реализация JWT состоит из двух токенов:
\begin{enumerate}
	\item{\textbf{Access Token} --- короткоживущий (5 минут) токен, содержит основные claims (Id, имя пользователя, роль).}
	\item{\textbf{Refresh Token} --- долгоживущий (24 часа) токен, содержит аналогичные claims, но с дополнительным Jti (уникальный идентификатор).}
\end{enumerate}

Access Token используется для аутентификации пользователя и получения разрешения на доступ к защищенному ресурсу. Refresh Token используется для обновления Access Token в случае его устаревания. Так же для предотвращения использования украденных валидных Refresh Token, данный токен сохраняется в базе данных и удаляется при выходе пользователя из системы или повторном входе, что обеспечивает его недействительность и исключает возможность повторного применения после завершения сессии.

% NOTE: Генерация токенов:
\textbf{Генерация токенов:} По данным пользователя генерируются токены с учетом их времени жизни и соответствующими секретным ключам.
\begin{minted}{cs}
    private string GenerateAccessToken(User user, string secretKey)
    {
        var claims = GenerateClaimsAccess(user);
        var token = GenerateToken(claims, secretKey, accessTokenTime);
        return new JwtSecurityTokenHandler().WriteToken(token);
    }

    private string GenerateRefreshToken(User user, string secretKey)
    {
        var claims = GenerateClaimsRefresh(user);
        var token = GenerateToken(claims, secretKey + "sault", refreshTokenTime);
        return new JwtSecurityTokenHandler().WriteToken(token);
    }
\end{minted}
% \textbf{Подпись токенов:}
% \begin{enumerate}
% 	\item{Access Token подписывается с помощью ключа secretKey хранящегося в appsettings.json.}
% 	\item{Refresh Token подписывается с помощью того же ключа secretKey + строки "sault".}
% \end{enumerate}

% NOTE: Валидация Refresh Token 
\textbf{Валидация Refresh Token:} Если Refresh Token валиден, то возвращает Ok, если просрочен, то Ok с описанием, что время жизни токена истекло, иначе BadRequest с описанием, что токен невалидный.
\begin{minted}{cs}
    private BaseResponse<Tokens> ValidateRefreshToken(string token, string secretKey)
    {
        var tokenHandler = new JwtSecurityTokenHandler();
        var validationParameters = new TokenValidationParameters
        {
            ValidateIssuerSigningKey = true,
            IssuerSigningKey = new SymmetricSecurityKey(Encoding.ASCII.GetBytes(secretKey + "sault")),
            ValidateIssuer = false,
            ValidateAudience = false,
            ClockSkew = TimeSpan.Zero
        };
        try
        {
            // Проверка подписи JWT, если токен не верный будет исключение
            tokenHandler.ValidateToken(token, validationParameters, out _);
            return BaseResponse<Tokens>.Ok();
        }
        catch (SecurityTokenExpiredException ex)
        {
            return BaseResponse<Tokens>.Ok(description: "Token expired");
        }
        catch (SecurityTokenException ex)
        {
            return BaseResponse<Tokens>.BadRequest(description: "Tokens not valid");
        }
    }
\end{minted}

% NOTE: Обновление токенов:
\textbf{Обновление токенов:} Обновление токенов состоит из нескольких этапов:
\begin{itemize}
	\item{Проверка токена на валидность;}
\end{itemize}
\begin{minted}{cs}
    public async Task<IBaseResponse<Tokens>> RefreshToken(string oldRefreshToken, string secretKey)
    {
        BaseResponse<Tokens> response = ValidateRefreshToken(oldRefreshToken, secretKey);

        // Если oldRefreshToken невалидный
        if (response.StatusCode == StatusCodes.BadRequest)
        {
            return response;
        }
\end{minted}

\begin{itemize}
	\item{Перенос данных из Refresh Token в User для дальнейшего создания Access Token по этим данным:}
\end{itemize}
\begin{minted}{cs}
        var oldToken = new JwtSecurityTokenHandler().ReadJwtToken(oldRefreshToken);
        User user = new User();
        // Заполняем user на основе данных из oldRefreshToken
        user.Id = Convert.ToInt32(oldToken.Claims.First(
                    claim => claim.Type == JwtRegisteredClaimNames.Sub
                    ).Value);
        user.Name = Convert.ToString(oldToken.Claims.First(
                    claim => claim.Type == JwtRegisteredClaimNames.Name
                    ).Value);
        user.Role = Convert.ToString(oldToken.Claims.First(
                    claim => claim.Type == ClaimTypes.Role
                    ).Value);
\end{minted}

\begin{itemize}
	\item{Проверка на нахождение Refresh Token в базе данных (если его там нет, то считаем недействительным):}
\end{itemize}
\begin{minted}{cs}
        // Находим oldRefreshToken
        var oldRefreshDB = await _RefreshTokenRepository.FirstOrDefaultAsync(token => token.Token == oldRefreshToken);
        if (oldRefreshDB == null)
        {
            response = BaseResponse<Tokens>.NotFound();
            return response;
        }
\end{minted}

\begin{itemize}
	\item{Обновление Refresh Token, если токен не просрочен:}
\end{itemize}
\begin{minted}{cs}
        var oldTokenDB = new JwtSecurityTokenHandler().ReadJwtToken(oldRefreshDB.Token);

        // Оставляем старый refreshToken, или в случае просрока заменим далее
        string refreshToken = oldRefreshDB.Token;
        if (response.Description == "Token expired")
        {
            // Удаляем oldRefreshToken, если тот истек
            await _RefreshTokenRepository.Delete(oldRefreshDB);
            // Пересоздаем oldRefreshToken, если тот истек
            refreshToken = GenerateRefreshToken(user, secretKey);

            RefreshToken saveRefreshToken = new RefreshToken
            {
                Id = user.Id,
                Token = refreshToken
            };
            // Сохраняем новый refreshToken в бд
            await _RefreshTokenRepository.Create(saveRefreshToken);
        }
\end{minted}

\begin{itemize}
	\item{Создание Access Token по данным извлеченным из Refresh Token:}
\end{itemize}
\begin{minted}{cs}
        string accessToken = GenerateAccessToken(user, secretKey);
        response = BaseResponse<Tokens>.Ok(new Tokens
        {
            AccessToken = accessToken,
            RefreshToken = refreshToken
        });
        return response;
    }
\end{minted}

% NOTE: Удаление Refresh Token из бд: 
\textbf{Удаление Refresh Token из бд:} Удаление Refresh Token из базы данных по Id пользователя:
\begin{minted}{cs}
    public async Task<IBaseResponse> DeleteRefreshToken(int userId)
    {
        BaseResponse response;

        var refreshToken = await _RefreshTokenRepository.FirstOrDefaultAsync(token => token.Id == userId);

        if (refreshToken == null)
        {
            response = BaseResponse.NoContent();
            return response;
        }
        await _RefreshTokenRepository.Delete(refreshToken);

        response = BaseResponse.Ok();
        return response;
    }
\end{minted}
