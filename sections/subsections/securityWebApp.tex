\subsection{Безопасность веб-приложения}

В свете растущих угроз безопасности веб-сервисов особое внимание уделяется надёжным механизмам защиты пользовательских данных. Ключевыми элементами такой защиты являются процессы аутентификации и авторизации, обеспечивающие контроль доступа к ресурсам, а также безопасное хранение паролей с использованием современных методов хэширования. Рассмотрение этих аспектов позволит минимизировать риски несанкционированного доступа и защитить пользователей от компрометации их учётных записей.

\subsubsection{Аутентификация и авторизация (JWT)}

Подводка к JWT

Структура JWT:

Подпись JWT

Срок жизни

Ролевая модель

\subsubsection{Хеширование паролей}

Безопасное хранение паролей пользователей является фундаментальной задачей при разработке систем аутентификации. Существует два способа хранения паролей в базеданных: в чистом виде и в виде зашифрованных значений. 

Хранение в чистом виде --- самый простой способ, при котором пароль хранится точно так, как его ввел пользователь. Это не безопасно, так как в случае утечки данных, злоумышленник может сразу использовать этот пароль.

Более распространённым и безопасным методом является хранение хэшированных паролей. Хэш"=функция --- это функция, которая принимает произвольное количество входных данных и выдает выходные данные фиксированного размера. При этом хэширование является односторонней операцией: восстановить исходный пароль из хэша практически невозможно.

Использование хэш"=функций позволяет хранить не сами пароли, а их хэш"=суммы. В случае компрометации базы данных злоумышленник не сможет напрямую использовать хэши для доступа к аккаунтам. Для повышения безопасности к паролям добавляют случайные данные, называемые солью (salt), что предотвращает атаки с использованием радужных таблиц. Радужная таблица --- это предварительно вычисленный набор хэш"=значений. Эти таблицы позволяют злоумышленникам получать доступ к защищенным системам без подбора пароля.

 Хэш"=функция бывают обратимыми и необратимыми, однако для защиты паролей используют исключительно необратимые функции.

\textbf{Ключеве принципы хеширования паролей:}
\begin{itemize}
	\item{\textbf{Необратимость:} Хэш"=функции должны быть спроектированы так, чтобы обратное преобразование хэша в исходный пароль было невозможно}	
	\item{\textbf{Детерминированность:} Один и тот же пароль всегда генерирует идентичный хэш}
	\item{\textbf{ Соль (salt):} Случайные данные, добавляемые к паролю перед хэшированием для предотвращения атак радужными таблицами}
\end{itemize}
