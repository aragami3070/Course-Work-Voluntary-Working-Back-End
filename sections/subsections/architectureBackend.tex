\subsection{Клиент"=серверная архитектура}

\subsubsection{REST API}
В современных информационных системах эффективное взаимодействие между различными приложениями и сервисами является ключевым фактором успешной реализации бизнес-логики и обеспечения высокого уровня обслуживания пользователей. Одним из наиболее распространённых и универсальных подходов к организации такого взаимодействия является использование REST API.

Передача представления состояния (Representational State Transfer, сокращенно REST) --- архитектурный стиль, который подчеркивает масштабируемость взаимодей ствий компонентов, их независимое развертывание и универсальность интерфейсов \cite{RestAPI3}

REST API основан на модели связи клиент"=сервер, где клиент с помощью HTTP"=запросов на сервер получает информацию в формате JSON, XML, HTML и так далее. Вся информация на сервере представлена в виде ресурсов и подресурсов: пользователи приложения, загруженные файлы и прочее. У каждого ресурса есть свой уникальный идентификатор, ресурс имеет состояние, и клиент может получать или изменять состояние ресурса при помощи представлений (как уже упоминалось ранее, под представлением можно понимать JSON, XML, текст в определённом формате или что угодно, что позволяет нам понимать текущее состояние ресурса). При этом серверная часть никак не зависит от клиентской, клиентом веб"=сервиса может выступать браузер пользователя, мобильное приложение, другой сервер и так далее. \cite{RestAPI2}

В REST для манипулирования ресурсами используются стандартные HTTP"=методы:
\begin{itemize}
	\item{GET --- получение содержимого;}
	\item{POST --- создание содержимого;}
	\item{PUT --- обновление содержимого;}
	\item{DELETE --- удаление содержимого;}
\end{itemize}

Запросы в такой архитектуре являются самодостаточными, серверу при их обработке нет необходимости извлекать контекст приложения, поскольку клиент включает в запрос все необходимые данные, использую для этого заголовки и тело запроса. Такой подход повышает производительность, упрощает дизайн и реализацию серверных компонентов системы. \cite{RestAPI1}

Для REST API выделяются несколько основных принципов:
\begin{enumerate}
	\item{использование клиент"=серверной модели связи;}
	\item{использование стандартных HTTP"=методов;}
	\item{использование уникальных идентификаторов ресурсов;}
	\item{работа с данными через представления в формате JSON;}
	\item{отсутствие состояния на стороне сервера;}
\end{enumerate}

Соблюдение перечисленных принципов повышает надёжность и производительность приложения.

\subsubsection{CRUD}
В современных информационных системах эффективное управление данными является основой для обеспечения функциональности и надежности приложений. Методология CRUD, представляющая собой совокупность четырёх базовых операций --- создание (Create), чтение (Read), обновление (Update) и удаление (Delete) --- служит фундаментальной концепцией при проектировании и реализации систем хранения и обработки данных. 

В контексте разработки серверной части приложения CRUD операции часто реализуются с помощью REST API, через стандартные HTTP"=методы GET, POST, PUT и DELETE.

Как правило, CRUD операции реализуются вместе с правами доступа. Это означает, что для доступа к некоторым CRUD операций необходимо быть авторизованным пользователем, а для каких-то иметь роль администратора. Например, для удаления пользователя необходимо авторизоваться пользователем с ролью администратора.

\begin{minted}{cs}
				[Authorize(Roles = "Admin")] // Требуется роль администратора
        public async Task<IActionResult> Delete(int id)
        {
						// Реализация
        }
\end{minted}
