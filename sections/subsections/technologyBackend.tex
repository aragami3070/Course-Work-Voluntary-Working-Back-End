\subsection{Технологии backend-разработки на .NET, ASP.NET Core}

\subsubsection{Middleware}

\subsubsection{Встроенная поддержка DI(Dependency Injection)}
В ASP.NET Core есть встроенная поддержка внедрения зависимостей (Dependency Injection, DI). Она представляет собой механизм, обеспечивающий слабую связанность компонентов приложения и упрощающий управление зависимостями между ними. DI позволяет автоматически предоставлять необходимые объекты (сервисы) в классы, которые от них зависят, без необходимости создавать эти объекты вручную.

В ASP.NET Core внедрение зависимостей реализовано через встроенный контейнер служб, который управляет жизненным циклом и разрешением зависимостей. Сервисы регистрируются в контейнере, обычно в файле конфигурации приложения (например, Program.cs), с указанием времени их жизни --- Transient,  Scoped или Singleton. 

\begin{itemize}
	\item{Transient: Сервис создается при каждом запросе.}
	\item{Scoped: Сервис создается один раз для каждого запроса (подключения).}
	\item{Singleton: Сервис создается только один раз.}
\end{itemize}

При создании экземпляра класса, например контроллера, контейнер автоматически внедряет зарегистрированные зависимости через конструктор, что называется конструкторной инъекцией.
