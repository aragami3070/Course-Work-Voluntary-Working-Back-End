\subsection{Проектирование серверной части}

\subsubsection{Архитектура решения}

Архитектура серверной части проекта построена по принципу модульности, где каждый модуль (проект) отвечает за отдельный аспект системы:

\begin{itemize}
	\item{\textbf{Controllers}: Обрабатывает отправленные клиентом HTTP-запросы и возвращает ответы в формате JSON;}
	\item{\textbf{Services}: Реализует логику приложения;}
	\item{\textbf{Context}: Содержит настройки DbContext и инкапсулирует логику взаимодействия с базой данных (паттерн репозитории);}	
	\item{\textbf{Database}: Содержит сущности используемые для взаимодействия с базой данных;}
\end{itemize}

\subsubsection{Ролевая модель}
 
\begin{itemize}
	\item{Роль "стуеднт" --- пользователь, который может просматривать ленту запросов, записываться на запрос и отписываться от него. Также может просматривать и изменять данные своего аккаунта и получать запросы, на которые он записан.}
	\item{Роль "администратор" --- пользователь, который может все, что может "студент", а также имеет доступ к CRUD операциям на запросы и пользователей. Также может отметить запрос как выполненный с начислением очков за него.}
	\item{Роль "Dev" --- пользователь, который может все, что может "студент" и "администратор", а также имеет доступ к повышению до администратора студента и понижению до студента администратора.}
\end{itemize}

\subsubsection{Спецификация API}

\begin{itemize}
	\item{/api/AdminRequest/ --- endpoint-ы для администраторов Dev-ов по работе с запросами. Включает в себя CRUD для запросов, получение ленты запросов для администраторов и возможность отметить запрос как выполненный с начислением очков за него.}
	\item{/api/AdminUser/ --- endpoint-ы для администраторов Dev-ов по работе с пользователями. Включает в себя CRUD для пользователей.}
	\item{/api/Auth/ --- endpoint-ы для аутентификации пользователей. Включает в себя регистрацию, аутентификацию пользователей, обновление токена, отзыв токена и проверка своей роли (endpoint, возвращающий роль).}
	\item{/api/DevUser/ --- endpoint-ы для Dev-ов по работе с администраторами. Включает в себя возможность выдать роль администратора студенту и понизить администратора до студента.}
	\item{/api/StudentRequest/ --- endpoint-ы для студентов по работе с запросами. Включает в себя возможность получении ленты запросов для студента, записаться на запрос и отписаться с него.}
	\item{/api/User/ --- endpoint-ы для всех пользователей. Включает в себя получение информации о себе, изменение пароля и своих данных и получении информации о запросах, на которые он подписался.}
\end{itemize}

