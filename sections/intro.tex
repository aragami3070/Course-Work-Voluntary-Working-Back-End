\intro

В процессе обучения многие студенты и преподаватели участвуют в различных мероприятиях. Они имеют недостатки ручного управления, такие как: низкая прозрачность, высокая трудоемкость контроля и риск ошибок при фиксации результатов. Их автоматизация обеспечит централизованный мониторинг выполнения, минимизирует временные затраты участников процесса и повысит достоверность данных за счет исключения человеческого фактора. Актуальность темы заключается в создании приложения для решения данной проблемы.

Целью данной работы является разработка серверной части системы для приложения "Отработки" с REST API на .NET.

В соответствии с поставленной целью необходимо решить следующие задачи:
% TODO: минимум еще пару задач написать
\begin{itemize}
	\item{Изучить принципы разработки серверных приложений на .NET.}
	\item{Спроектировать архитектуру серверной части.}
	\item{Реализовать API для управления пользователями, запросами на участие в мероприятиях.}
	\item{Обеспечить безопасность (аутентификация, авторизация)}
\end{itemize}
